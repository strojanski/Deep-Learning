\documentclass{article}
\pdfpagewidth=8.5in
\pdfpageheight=11in
\usepackage{ijcai19}

% Use the postscript times font!
\usepackage{times}
\usepackage{soul}
\usepackage{url}
\usepackage[hidelinks]{hyperref}
\usepackage[utf8]{inputenc}
\usepackage[small]{caption}
\usepackage{graphicx}
\usepackage{amsmath}
\usepackage{booktabs}
\usepackage{multirow}

\urlstyle{same}

\title{Deep learning homework template}

\author{
Student name (ID)
}

\begin{document}

\maketitle

\section{Introduction}
A short report should be submitted for each homework. The reports should be up to 3 pages long but can be shorter as long as the questions in the tasks are properly answered. 

\section{Image data}
In some homeworks images should be added to display your qualitative results. This is necessary only in Homework 2.

\begin{figure}[!htb]
\centering
  \includegraphics[width=0.8\linewidth]{hw1_nn.pdf}
\caption{Example image.}
\label{fig:vqvae}
\end{figure}


\section{Tables}
You might want to show the results of various experiments in tables as these are the easiest to read.
\begin{table}[!htb]
\centering
\resizebox{1.0 \linewidth}{!}{\begin{tabular}{c c c c c c c}
\hline
% & &\multicolumn{5}{c}{Number of segmented training images}\\
Experiment & Schedule & L2 reg. & Optimizer & LR & Epoch & Cls. Acc.\\ \hline
Exp. 1  & None & Yes & Adam & 0.001 & 50 & 52\% \\
Exp. 2  & Step & No & SGD & 0.1 & 50 & 55\% \\
Exp. 3  & Exponent. & Yes & SGD & 0.2 & 50 & 42\% \\
Exp. 4  & None & Yes & Adam & 0.001 & 100 & 58\% \\
Exp. 5  & Step & Yes & Adam & 0.01 & 100 & 18\% \\
Exp. 6  & Exponent. & Yes & Adam & 0.1 & 200 & 28\% \\
Exp. 7  & None & Yes & Adam & 0.01 & 200 & 48\% \\
\hline
\end{tabular}}
\caption{Table of random experiments. Not actual results or suggestions. }
\label{tab:tab1}
\end{table}

You should also comment the achieved results in the text. Additionally, you should list the architecture of the networks used in the text or in a table. This is mostly relevant for HW1 and some tasks of HW3. 


\end{document}

